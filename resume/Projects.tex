\cvsection{Projects}

\begin{cventries}
	\cventry
	{} % Empty position
	{Zhejiang Health Insurance Payment Reform} % Project
	{Zhejiang University} % Empty location
	{May. 2022 - Jun. 2023} % Empty date
	{
		\quad The objective of the project is to cluster similar medical records based on the complete annual diagnosis records in Zhejiang Province and regulate doctors' diagnostic behavior by paying the same price for each type of similar medical record through medical insurance, avoiding unnecessary excessive medical behavior. \newline
		\quad From an algorithmic perspective, the essence of the project is a large-scale clustering algorithm based on real-world data. It encompasses various stages such as data cleaning, the FP-growth frequent pattern mining algorithm, the minibatch Kmeans clustering algorithm, distributed computing, and the utilization of a Slurm cluster. I am primarily responsible for developing the code components of the project, excluding tasks related to resource scheduling and cluster construction. \newline
		The main highlights of the project include: \newline
		\vspace{3.5mm}
		\begin{cvitems} % Description(s) bullet points
			\item{Leveraging the unique characteristics of the data to classify it based on disease types, thereby significantly reducing the amount of data processed in a single time and enabling efficient clustering.}
			\item{In the training phase, a simple cache calculation result technology is used, which greatly reduces the time for debugging the algorithm and makes it possible to try different ideas and technical details.}
			\item{Transforming the process of traversing the FP-tree in the prediction stage into a matrix multiplication-based similarity problem. This optimization makes the code memory friendly and enables acceleration through the utilization of matrix operation libraries.}
			\item {Through the combined effects of the second and third highlights, the algorithm's iteration cycle for training and forecasting has been dramatically reduced. Previously, the process necessitated the utilization of five machines running continuously for 30 days to handle merely one-fifth of the city's data. However, with the advancements made, the same task can now be completed using only one machine within approximately three days, encompassing the entirety of the city's dataset. This improvement yields a speedup of approximately 250 times.}
		\end{cvitems}
	}

	\cventry
	{} % Empty position
	{Cervical Lesion Cell Detection} % Project
	{Zhejiang University} % Empty location
	{Oct. 2020 - Mar. 2022} % Empty date
	{
		Automatic detection of cervical lesion cells or cell clumps using cervical cytology images is critical for efficient cervical cancer screening. I have studied this project through three aspects: \newline
		\vspace{4.5mm}
		\begin{cvitems} % Description(s) bullet points
			\item{The large appearance variances between single cell and cell clumps of the same lesion type pose difficulties for accurate lesion cell recognition.}
			\item{Normal cells can be a good reference for abnormal cell recognition.}
			\item{The visual similarity problem among certain abnormal cells, especially those in adjacent differentiated stages.}
		\end{cvitems}
		\vspace{4.5mm}
		For the above issues, I have studied a novel task decomposing and cell comparing framework for cervical lesion cell detection. The task decomposing scheme decomposes the original detection task into two detection subtasks, which encourages the network to focus on specific cell structures. The cell comparing scheme imitates clinicians to utilize normal cells as references and compare different types of abnormal cells, and it allows the model to learn more effective and useful lesion cell features.
	}

	\cventry
	{} % Empty position
	{The 12$^\text{th}$ World Robotics Sailing Championship (WRSC)} % Project
	{Zhejiang University} % Empty location
	{Oct. 2018 - Aug. 2019} % Empty date
	{
		\quad In this championship, each team needs to design an autonomous sailing robot that can complete the specified tasks. I'm the leader of the software team, and responsible for the development of the decision system and vision system. Our team wins the Champion of the 12$^\text{th}$ WRSC. Our code is avilible at \href{https://github.com/ZMART-Sailing/sailing_robot}{\textcolor{link}{[Code]}}.\newline
		My main contributions are as follows: \newline
		\vspace{3.5mm}
		\begin{cvitems} % Description(s) bullet points
			\item{Developing a rule-based decision system, including a path planning module and an obstacle avoidance module. It outputs the desired rudder angle and sail angle of the boat based on the position information and the wind information.}
			\item{Developing the vision system, including an obstacle detection module and a QR-code scan module. They are deployed on an Nvidia Jetson Nano.}
		\end{cvitems}
	}

	\cventry
	{} % Empty position
	{Contributions to Open Source Projects} % Project
	{\hfill} % Empty location
	{} % Empty date
	{
		\vspace{-3.5mm}
		\begin{cvitems} % Description(s) bullet points
			% I have the ability to dig into the source code of a large project.
			\item{I have the skill to explore and analyze the source code of large projects. For examples, I contributed to \href{https://github.com/Lightning-AI/lightning}{\textcolor{link}{Pytorch-lightning}}, mmrepos including \href{https://github.com/open-mmlab/mmdetection}{\textcolor{link}{mmdetection}}, \href{https://github.com/open-mmlab/mmengine}{\textcolor{link}{mmengine}} and \href{https://github.com/open-mmlab/mmcv}{\textcolor{link}{mmcv}}, and so on.}
			\item{I have developed a boilerplate project called \href{https://github.com/shenmishajing/project_template}{\textcolor{link}{project\_template}}, which is built on PyTorch Lightning. This project incorporates a wide range of common engineering features, such as config file inheritance, cross-validation, hyperparameter tuning via wandb, and all other features provided by PyTorch Lightning. By utilizing this project, we can concentrate on model implementation without being burdened by the intricacies of engineering details. Some examples: \href{https://github.com/shenmishajing/mmdet_lightning}{\textcolor{link}{mmdet\_lightning}}, \href{https://github.com/shenmishajing/mmpretrain_lightning}{\textcolor{link}{mmpretrain\_lightning}} and \href{https://github.com/shenmishajing/detectron2_lightning}{\textcolor{link}{detectron2\_lightning}}.}
		\end{cvitems}
	}
\end{cventries}
